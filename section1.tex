\section{Assemblies}

A \emph{($\catset$)-indexed poset} is a contravariant functor
$\fifd:\setpos$ from the category of sets to the category $\catpos$ of
posets and monotone maps. For $I\in\catset$, the poset
$\fifd(I)$ is called the \emph{fiber} of $\fifd$ over $I$ and its elements are
called \emph{predicates} on $I$. The monotone maps
$\triq(f):\triq(I)\to\triq(J)$ for $f:J\to I$ are called \emph{reindexing maps}
and are abbreviated $f^*$. The \emph{total category} (\aka \emph{Grothendieck
construction}) of $\fifd$ is the category $\gcons\fifd$ whose objects are pairs
$(I\in\catset,\varphi\in\fifd(I))$ and whose morphisms from $(I,\varphi)$ to
$(J,\psi)$ are functions $f:I\to J$ satisfying $\varphi\leq f^*\psi$.
The total category admits a forgetful functor $\gcons\fifd\to\catset$ which is a
Grothendieck fibration.

% Given indexed posets $\fifd$ and $\fifb$, an \emph{indexed monotone map}
% is a natural transformation $\Phi:\fifd\to\fifb$.

% An \emph{indexed meet-semilattice} is an indexed poset 
% $\fifd:\setpos$ 
% where all
% fibers are meet-semilattices and all reindexing maps preserve finite
% meets\footnote{This is equivalent to $\gcons\fifd$ having, and
% $\gcons\fifd\to\catset$ preserving finite limits.}. An indexed monotone map
% % $\Phi:\fifd\to\fifb:\setpos$ 
% between \imsls is called \emph{cartesian} if it
% preserves fiberwise finite meets. 
% % If $\catc$ has 
% % finite limits then $\Phi$ is cartesian \emph{iff} $\gcons\Phi$ preserves 
% % finite products \emph{iff} $\gcons\Phi$ preserves finite limits.

A \emph{first-order hyperdoctrine} is an indexed poset $\triq:\setpos$
satisfying the following conditions.
\begin{description}
\item[(HA)\label{ax:ha}] All fibers $\triq(I)$ are {Heyting algebras}, and
all reindexing map preserve Heyting algebra structure.
    \item[(Q)\label{ax:q}] 
All reindexing maps $f^*$  have left and right adjoints $\exists_f\adj f^*
\adj \forall_f$, and we have 
    $\exists_k\circ h^*= g^* \circ\exists_f$ and
    $\forall_k\circ h^*= g^* \circ\forall_f$ for all $\catset$-pullbacks
\begin{equation}
\begin{tikzcd}
        D\ar[r, "k" description]\ar[d,"h"'] & B\ar[d,"g"]\\
        A\ar[r,"f" description] & C
\end{tikzcd}.
\end{equation}
\end{description}
A \emph{generic predicate}\footnote{Andrew used `weakly' recently, but in the
realizability literature they are simply called `generic'.} in an indexed poset
$\fifd$ is a predicate $\iota$ on a set $\prop$ such that for every set $I$ and
predicate $\varphi\in\fifd(I)$ there exists a (not necessarily unique)
$f:J\to\prop$ with $\varphi=f^*\iota$. In other words, a generic predicate is a
weakly terminal object in the wide subcategory of $\gcons\fifd$ on cartesian
maps. 

\begin{definition}
A \emph{tripos} is a \foh which admits a
generic predicate.
\end{definition}
\begin{examples}
\begin{enumerate}
\item For every frame $A$, the indexed preorder
\begin{equation}
\trip_A :\setpos\qquad \trip_A(I)=(A^I,\text{pointwise ordering})
\end{equation}
We call such triposes
\emph{localic}.
\item 
Given a frame $A$ and a filter $F\subs A$, we can define a `filter quotient'
$\trip_A/F$ by setting $(\trip_A/F)(I)$ the poset-reflection of the 
preorder $(A^I,\leq_F)$ where 
$h\leq k:\Leftrightarrow\bigwedge_i(hi\Rightarrow ki)\in F$.
Such filter quotients are generally not localic.
\end{enumerate}
\end{examples}



An indexed monotone map $\Phi:\triq\to\triq$ between triposes is called
\emph{regular}, if it preserves fiberwise finite meets and satisfies
$\ex_f\circ\Phi_J=\Phi_I\circ\ex_f$ for all $f:J\to I$. Regular maps between
localic triposes correspond uniquely to frame morphisms.

A predicate $\varpi\in\triq(I)$ of a tripos is
called \emph{weakly \eprime}, if for every function
% $I\xleftarrow{u}J$ 
$u: J \to I$ and every $\psi\in\triq(J)$ satisfying $\varpi\leq\exists_u\psi$,
there exists a section $s$ of $u$ such that $\varpi\leq
s^*\psi$\footnote{Equivalently, $\varpi$ is prime if $(I,\varpi)$ has the left
lifting property with respect to all cocartesian maps in $\gcons\triq$}. A
predicate is called \emph{\eprime}, if all of its reindexings are weakly
\eprime. Since \eprime predicates are closed under reindexing by definition,
they form a indexed sub-poset $\eprim(\triq)\subs\triq$.

A predicate $\delta\in\triq(I)$ is called \emph{discrete}, if for every span
$J\xepil{e}K\xrightarrow{f}I$ with $e$ surjective, and every $\varphi\in\fifd_J$
such that $e^*\varphi\leq f^*\mu$, there exists a (necessarily unique) $h:J\to
I$ such that $he=f$ and $\varphi\leq h^*\mu$. In other words, $\delta$ is
discrete if $(I,\delta)$ is right orthogonal in $\gcons\triq$ to all cartesian
maps over surjections. We call $(I,\delta)\in\gcons\triq$ a \emph{discrete
object} whenever $\delta$ is a discrete predicate.

\begin{definition} $ $
\begin{enumerate}
\item 
A \emph{relative realizability tripos} is a first-order hyperdoctrine $\triq$
such that 
\begin{enumerate}
\item $\eprim(\triq)$ is closed under finite meets in $\triq$, 
\item $\eprim(\triq)$ admits a discrete generic predicate, and 
\item for every $\varphi\in\triq(I)$ there exists a function $f:J\to
I$ and an \eprime $\varpi\in\triq(J)$ with $\varphi=\ex_f\varpi$.
\end{enumerate}
\item A \emph{realizability tripos} is a relative realizability tripos $\triq$
such that 
$\eprim(\triq)(1)=1$.
\end{enumerate}
\end{definition}
The terminology is justified by the following lemma.
\begin{lemma}
Every relative realizability tripos is a tripos.
\end{lemma}
\begin{proof}
A generic predicate of $\triq$ is given by $\exists_q(p^* \iota)$, where 
$\iota\in\eprim(\triq)(A)$ is a generic predicate of $\eprim(\triq)$, and 
$\langle p,q\rangle:E\incl A\times PA$ is the membership relation between 
$A$ and its powerset.
\end{proof}
The tripos $P:\setpos$ maps each set $I$ to its powerset $P(I)$ ordered by 
inclusion.
Every \foh $\triq$ admits \fmp indexed monotone maps
\begin{align}
\gamma&:\triq\to P &\gamma_I(\varphi)&=\setof{i\in I}{i^*\varphi=\top}
\quad\text{and}\\
\delta&: P \to\triq& \delta_I(U)&= \exists_{U\incl I}\top.
\end{align}
\begin{lemma}
A tripos $\triq$ is localic iff $\delta$ is left adjoint to $\gamma$.
\end{lemma}
\begin{proof}
Exercise.
\end{proof}
We call a \foh $\triq$ \emph{totally connected}, if $\delta:P\to\triq$ 
has a \fmp left adjoint.
\begin{lemma}
For any totally connected \foh $\triq$, the indexed closure operator
$\delta\circ\pi:\triq\to\triq$ coincides with double negation.
\end{lemma}
\begin{proof}
\cite[3.4.34]{frey2013fibrational}.
\end{proof}
\begin{lemma}
Relative \rtrips are totally connected.
\end{lemma}
\begin{proof}
Let $\varphi\in\triq(I)$ and let $f:J\to I$ and $\varpi\in\eprim(\triq)(J)$ such
that $\varphi = \exists_f\varpi$. Then $\pi_I(\varphi)=\image(f)$ (although $f$
and $\varpi$ are not uniquely determined by $\varphi$ in general, $\image(f)$
is). It is easy to see that this defines left adjoints $\pi_I$ to $\delta_I$ 
for all sets $I$, and furthermore that the $\delta_I$ are natural in $I$.
\end{proof}
\begin{lemma}
We have $\gamma\adj\delta$ for realizability triposes.
\end{lemma}
\begin{proof}
By the previous lemma it is sufficient to show that $\pi=\gamma:\triq\to P$
whenever $\eprim(\triq)(1)=1$. Let $\varphi\in\triq(I)$ and $i\in I$. We have
$i\in \pi_I(\varphi)$ iff $\pi_1(i^i\varphi)=\top$ by pullback stability, and
$i\in\gamma_I(\varphi)$ iff $i^*\varphi=\top$ by definition. Thus, it is enough
to show that $\pi_1$ reflects $\top$. Assume that $\pi_1(\psi)=\top$. Then there
exists a surjective $e:J\to 1$ and an \eprime $\varpi$ on $J$ such that
$\exists_e\varpi=\psi$. For any section $s$ of $e$ we have $\psi =
\exists_e\varpi \geq s^* \varpi = \top$, where the latter equation follows from
$\eprim(\triq)(1)=1$.
\end{proof}

\begin{definition}
Let $\triq$ be a totally connected tripos. The category $\catasm(\triq)$ is 
defined to be the full subcategory of $\gcons\triq$ on objects $(I,\varphi)$
such that $\varphi$ is dense for the indexed closure operator 
$\delta\circ\pi:\triq\to\triq$.
\end{definition}


TODO: 
\begin{itemize}
\item $\catasm$ as a subcategory of the topos
\item $\catasm$ is a quasitopos with a generic mono
\item next problem: how to describe internal simplex/cube category? do we want 
to allow ourselves extensional type theory as internal language of asm?
\item further: internal presheaves
\end{itemize}